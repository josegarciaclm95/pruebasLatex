\documentclass[11pt,a4paper]{article}

\usepackage[spanish]{babel} %babel - paquete de los idiomas 
\usepackage[utf8x]{inputenc} % dios te bendiga (para las tildes)
\usepackage[T1]{fontenc} 
\usepackage[left=4cm,right=3cm,top=4cm,bottom=3cm]{geometry}
\usepackage{fourier} % fuente más mejor que la original
% Autor y documento
\author{José María García García}
\title{Artículo 2: The Critical Success Factors Across ERP Implementation Processes}

\begin{document}
\maketitle
\begin{abstract}
Leer el artículo y obtener de el información acerca de los siguientes puntos (leer hasta el comienzo de la sección Five Dimensions of Chinese culture).
\end{abstract}
%
% en section, las tildes se ponen \'vocal. \'a, \'e, etc.
%
\section{Preguntas}
% Primera pregunta.
\begin{itemize}
\item \textbf{¿Cómo es el proceso de implantación de un ERP?}
\end{itemize}
La implementación de un ERP constituye un proyecto grande y complejo que implica a gran cantidad de gente y recursos trabajando juntos, contrarreloj y haciendo frente a problemas inesperados. 
%Se han estudiado los factores críticos para el éxito pero no se le ha prestado atención a los modelos de implementación.

% Segunda pregunta. 
\begin{itemize}
\item \textbf{¿En qué consiste la metodología de los CSF (Critical Success Factors)?}
\end{itemize}
Los CSFs, que podríamos traducir al español como \textbf{factores críticos para el éxito} (de la implementación), fueron concebidos como una herramienta para identificar aquello que una empresa tenía que hacer bien para triunfar en la implementación. La metodología de los CSF consiste en asegurar la presencia de esos factores a la hora de llevar a cabo la implementación.

% Tercera pregunta
\begin{itemize}
\item \textbf{¿Según Somers y Nelson, cuales son los 5 CSF principales en una implementación ERP?}
\end{itemize}
Somers y Nelson estudiaron más de 110 implementaciones de ERPs, de cuyo estudio obtuvieron una lista con 22 CSFs. Los cinco más importantes son el \textbf{apoyo de los directivos}, \textbf{la capacidad del equipo de proyecto}, \textbf{la cooperación entre departamentos}, \textbf{tener metas y objetivos claros} y  \textbf{ la gestión del proyecto}.

% Cuarta pregunta. 
\begin{itemize}
\item \textbf{¿Y según el estudio de Nah, Zuckweiler y Lau de 2003?}
\end{itemize}
Nah, Lau, y Kuang llevaron a cabo un estudio similar del cual extrajeron que 11 era el número de factores críticos para el proceso de implementación. De esos once, los cinco más importantes eran el \textbf{apoyo de los directivos}, \textbf{la defensa del proyecto}, \textbf{el trabajo del equipo}, \textbf{la gestión del proyecto} y  \textbf{cambiar la forma en la que se realiza la gestión en la empresa}.
% Quinta pregunta
\begin{itemize}
\item \textbf{¿Hay un consenso acerca de cuales son los CSF de la implementación de un ERP?}
\end{itemize}
No existe un consenso que permita definir cuales son los factores críticos para el éxito de la implementación ya que estos factores existen en un entorno complejo con un gran numero de interrelaciones, además de ser completamente subjetivos. Debido a la naturaleza dinámica del proceso de implementación, los factores que pueden ser críticos en un momento dado, pueden no serlo al momento siguiente. No obstante, son varios los estudios que coinciden en que el apoyo de las unidades de gestión superiores, la gestión del proyecto, el trabajo en equipo, y la reeducación son factores críticos para el éxito.

%sexta pregunta
\begin{itemize}
\item \textbf{¿Cuáles son los nueve pasos de la implementación propuesta? ¿En qué consiste cada uno de ellos?}
\begin{enumerate}
%\setcounter{enumi}{4} para empezar enumeracion desde un punto determinado.
\item \textbf{Planificación de la implementación.} 
Paso crucial ya que durante el mismo se forman expectativas y se crea una buena relación de trabajo con el cliente. El consultor ha de mostrar confianza y conocimiento de la metodología que se aplique (en el caso del artículo, FOCUS). 

\item \textbf{Planificación de la educación.} 
Durante este paso, entrenamos al equipo de implementación. Los miembros del mismo han de tener claro el propósito y lo que se espera de la implementación. Tras esto, el equipo tendrá una buena base para entender el sistema y hacer aportaciones útiles.

\item \textbf{Negocio piloto.}
El equipo se empieza a preguntar cómo se desarrollará la actividad de la empresa cuando el ERP esté implementado. El equipo está emocionado, pero también asustado, por miedo a no estar a la altura del proceso de implementación. Se lleva a cabo una simulación de cómo será el negocio una vez que esté en marcha el ERP dirigida por el consultor. El proceso de simulación implica los siguientes pasos: planificar las sesiones de la simulación, planificar la simulación en sí, ejecutar la simulación y por último, tratar temas sin concluir y/o problemas que puedan surgir.

\item \textbf{Desarrollo de procedimientos operacionales.}
Los procedimientos constituyen un esquema del flujo de negocio, por lo que aportan cierta estructura y control sobre procesos individuales. Se suelen desarrollar en base a gráficos sobre el flujo del trabajo. Para empezar a escribir los procedimientos, plasmamos la empresa (y sus departamentos) tal y como queda tras el proceso de integración (para esto se utilizan los mapas sobre el flujo de trabajo). La construcción del proceso no comienza hasta que no terminan los procesos de negocio del paso 3 (negocio piloto). Este, la escritura de procedimientos, es el proceso que más tiempo consume. 

\item \textbf{Conversión de datos.} 
Se lleva a cabo un mapeo de los datos y una programación de conversión para adaptar los datos antiguos al nuevo sistema. Previamente, se construyen especificaciones para los programas de conversión de datos, tanto para aprovechar datos antiguos como para reducir el esfuerzo de introducir datos en el futuro.

\item \textbf{Planificación de la migración.} 
Este paso asegura que el sistema final es completamente funcional y satisface los requisitos del cliente. Esta planificación contiene los pasos que será necesario llevar a cabo, el tiempo que va a llevar su ejecución y las responsabilidades de cada participante. Estos pasos se conocen a través de una reunión del consultor y el equipo de implementación. 

\item \textbf{Entrenamiento del usuario final.} 
Este paso (junto con el anterior) constituye el último paso de preparación previo a la puesta en marcha del sistema. El usuario final ha de ser entrenado, para lo que se diseña un plan de migración que asegura una transición correcta desde el sistema de prueba al sistema que tenemos justo antes de ponerlo en marcha. Estos dos procesos (entrenamiento y planificación del mismo) requieren iguales cantidades de esfuerzo para que el inicio del sistema sea correcto. 

\item \textbf{Asistencia para la puesta en marcha.} 
Es necesario un servicio técnico que pueda ayudar frente a problemas que surjan durante la puesta en marcha que se puedan resolver rápida y firmemente.

\item \textbf{Revisión tras la implementación.}
Este paso da lugar a un informe completo y formal sobre la implementación del sistema por la que el cliente ha pagado. Durante este paso, se evalúan las actividades finalizadas y se da cierre a aquellas que están sin terminar. Dura entre 30 y 60 días.
\end{enumerate}
\end{itemize}

\begin{itemize}
\item \textbf{¿Qué CSF es el más importante en cada uno de los pasos y por qué?}
\end{itemize}
\textbf{Paso 1 y 3. Gestión de las expectativas.} La gestión de las expectativas resulta muy importante en las primeras etapas (siendo el CSF más importante en los pasos 1 y 3) ya que los clientes pueden tener unas expectativas que no se corresponden con la realidad (pensar que el sistema aportará más beneficios de los que dará en realidad, o que será más fácil de instalar de lo que realmente es). Es tarea del consultor hacerles participar para que vean lo que supondrá el sistema realmente. Al llegar al paso 4, ya se puede apreciar todo el sistema, por lo que la gestión de expectativas deja de ser importante de ahí en adelante. 

\textbf{Paso 2 y 9. Uso de consultores.} Los miembros de la empresa cliente no son expertos en la implementación de sistemas, por lo que se hace necesario una figura externa (la figura de consultor) para ayudar en el proyecto. Esta figura puede dedicarse simplemente a resolver dudas (proveer conocimiento) o ejercer el rol de coordinador. Un consultor tiene conocimientos y experiencia como para controlar todos los pasos del proceso. En el paso 1 nos ofrecen su experiencia y elaboran el programa del proyecto. En el paso 2, diseñan las estrategias de educación. En el paso 3, aseguran una correcta ejecución de las estrategias piloto y de la simulación de actividades. En los pasos 8 y 9, ayudan en la resolución de problemas.

\textbf{Paso 4 y 6. Cooperación entre departamentos.} Resulta crucial en los pasos 3, 4 y 6, ya que son las etapas más cooperativas. En las primeras, los departamentos cooperan para llevar a cabo la simulación de actividades (paso 3) y para escribir los procedimientos de operación (paso 4), actividad que requiere de la participación de todos los departamentos, mientras que en la última (paso 6) cooperan para asegurar que el ERP funciona sin problemas.

\textbf{Paso 5. Análisis de datos y conversión.} Es una de las actividades más complejas y arriesgadas del proceso. Un problema en esta fase puede suponer un retraso importante en todo el proceso de implementación.

\textbf{Paso 7. Implicación del usuario.} En otras palabras, su participación en el proceso de implementación. Importante también en los pasos 4 y 8. Es importante en el 4 porque participar en la escritura de los procedimientos operacionales les permite entender mejor los nuevos procesos de negocio así como de lo que es capaz de hacer el sistema. Su participación puede ayudar también a desvelar fallos en etapas tempranas. El hecho de que los usuarios conozcan el sistema supondrá una ventaja cuando este se ponga en marcha. 

\textbf{Paso 8. Recursos dedicados.} Si intentamos llevar a cabo una compartición de los recursos en los pasos 6 y 8, pueden surgir problemas que retrasen el proyecto. Las actividades relacionadas con la puesta en marcha han de tener sus propios recursos.


\end{document}