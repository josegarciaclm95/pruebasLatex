\documentclass[11pt,a4paper]{article}

\usepackage[spanish]{babel} %babel - paquete de los idiomas 
\usepackage[utf8x]{inputenc} % dios te bendiga (para las tildes)
\usepackage[T1]{fontenc}


\usepackage[left=4cm,right=3cm,top=4cm,bottom=3cm]{geometry}

\author{José María García García}
\title{Artículo 2: The Critical Success Factors Across ERP Implementation Processes}


\begin{document}

\maketitle
\begin{abstract}
Leer el artículo y obtener de el información acerca de los siguientes puntos (leer hasta el comienzo de la sección Five Dimensions of Chinese culture).
\end{abstract}
% en section, las tildes se ponen \'vocal. \'a, \'e, etc.

\section{Preguntas}

\begin{enumerate}
\item ¿Cómo es el proceso de implantación de un ERP?
\item ¿En qué consiste la metodología de los CSF (Critical Success Factors)?
\item ¿Según Somers y Nelson, cuales son los 5 CSF principales en una implementación ERP?.
\item ¿Y según el estudio de Nah, Zuckweiler y Lau de 2003?
\item ¿Hay un consenso acerca de cuales son los CSF de la implementación de un ERP?
\item ¿Cuáles son los nueve pasos de la implementación propuesta? ¿ en qué consiste cada uno de ellos?
\item ¿Qué CSF es el más importante en cada uno de los pasos y por qué?
\end{enumerate}

\end{document}